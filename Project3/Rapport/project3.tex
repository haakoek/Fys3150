\documentclass[a4paper,10pt]{article}
\usepackage[utf8]{inputenc}
\usepackage{amsmath}
\usepackage{graphicx}
\usepackage{amssymb}
\usepackage{hyperref}
\usepackage{caption}
\usepackage{subcaption}
\usepackage{amsfonts}
\usepackage{enumerate}
\usepackage{wasysym}
\usepackage{algorithm2e}
\LinesNumbered % Number algorithm lines

%opening
\title{Project3 Fys3150}
\author{Håkon Kristiansen and Stian Goplen}

\begin{document}

\maketitle

\begin{abstract}

\end{abstract}

\section{Introduction}
In this project we will implement a model called Molecular Dynamics (MD). MD allows us to study the dynamics of atoms
and explore the phase space. The atoms interact through a force given as the negative gradient of a potential function. In this
project we use the Lennard-Jones potential. With this force we can integrate Newton's laws. While exploring the phase space, we will
sample statistical properties such as energy, temperature, pressure and heat capacity. Further we will look at phase transitions.

Initially we where given a C$++$ code skeleton containing the basic features of an MD code like the possibility to create atoms, a fully 
functional class for three-dimensional vectors, etc.
The program creates 100 argon atoms and place them uniformally inside a box of 10 Angstroms (Å). Each atom is given a velocity according
to the Maxwell-Boltzmann distribution. The program evolves the system in time with no forces so that all atoms move in a straight line. It creates
a file called movie.xyz containing all timesteps so it can be visualized with Ovito. Throughout the whole project we study the 
properties of argon. 

\section{Modifications To the Code}
\subsection{Periodic Boundary Conditions}
The first modifaction we do to the code is that we want to simulate a system of infinte size, eliminating boundary effects. In order to do so
we apply periodic boundary conditions. We restrict ourselves to a square system so that $L_x = L_y = L_z = L$ with $L$ being the
length of the box. Periodic boundary conditions ensures that if a atoms leaves the box in any direction we give it a position keeping it
in the box. It is implented in the function \textit{applyPeriodicBoundaryConditions} in the \textit{System} class. For the $x$-direction we have the following
algorithm
\begin{algorithm}[H]
  \caption{Periodic Boundary Conditions}
  \SetAlgoLined
  Assume that $r_x$ is the position of the atom in the $x$-direction\; \label{first_step}
  \uIf{\text{$r_x < 0$}}{
    $r_x = r_x + L_x$\;
    }
 \ElseIf{\text{$r_x \geq L_x$}} {
     $r_x = r_x - L_x$
   }
\end{algorithm}

\vspace{3mm}

\noindent which is done exactly the same way in the $y$ and $z$-direction.

\end{document}
